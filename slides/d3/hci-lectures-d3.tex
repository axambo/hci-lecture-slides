\documentclass[screen, aspectratio=43]{beamer}
\usepackage[T1]{fontenc}
\usepackage[utf8]{inputenc}

% Use the NTNU-temaet for beamer 
% \usetheme[style=ntnu|simple|vertical|horizontal, 
%     language=bm|nn|en, 
%     smalltitle, 
%     city=all|trondheim|alesund|gjovik]{ntnu2017}
\usetheme[style=ntnu,language=en]{ntnu2017}

\usepackage[english]{babel}
\usepackage[style=numeric,backend=biber,natbib=false,sorting=none]{biblatex}

\title[HCI-intro]{Human Computer Interaction}
\subtitle{Introduction}
\author[A. Xamb{\'o}]{Anna Xamb{\'o}}
\institute[NTNU]{Department of Music, NTNU}
\date{30 October 2018}
%\date{} % To have an empty date

\addbibresource{../hci-lectures.bib} % Add bibliography database

% Set the reference style to numeric.
% See here: http://tex.stackexchange.com/questions/68080/beamer-bibliography-icon
\setbeamertemplate{bibliography item}[text] 

% Set bibliography fonts to a small size.
\renewcommand*{\bibfont}{\footnotesize}

\begin{document}

\begin{frame}
  \titlepage
\end{frame}

% Alternatively, special title page command to get a different background
% \ntnutitlepage
\begin{frame}
\frametitle{Comments on the last two assignments}
\begin{itemize}
\item HCI Individual Assignment Day 2
\item HCI Group Assignment Day 2
\item Any other comments?
\end{itemize}
\end{frame}
%
\begin{frame}
\frametitle{Learning Outcomes}
\begin{itemize}
\item Get a sense of the synergies between HCI and NIME research.
\item Explore a range of key practices in the NIME community from an HCI perspective (design \& evaluation).
\item Identify the NIME practices relevant to personal projects.
\item Discern the format of NIME paper writing.
\end{itemize}
\end{frame}
%
\begin{frame}
\frametitle{Class Structure}
\begin{itemize}
\item 10.15-10.20 Comments on the two last assignments.
\item 10.20-10.40 Mapping the NIME field.
\item 10.40-11.00 Presentation/lecture of a selection of practices in NIME.
\item 111.00-11.30 Team work: Understanding NIME paper writing.
\item 11.30-12.00 The teams summarize to the group their selected paper (10 min per group). 
\end{itemize}
\end{frame}
%
\begin{frame}
\frametitle{Preparation: Mindmap}
\begin{itemize}
\item Create an intuitive mindmap based on a brainstorming session with your team about the topics from NIME and HCI that you think are related to the prototype that you built during the physical computing workshop.
\end{itemize}
\end{frame}
%
\begin{frame}
\frametitle{Mapping the NIME field}
\begin{itemize}
\item Presentations and discussion about the mindmaps:
\begin{itemize}
\item What was the workflow to create the mindmap? (e.g. from paper to digital, Chinese whispers styles, etc)
\item What tool(s) have you used?
\item How did you come across with the concepts?
\end{itemize}
\end{itemize}
\end{frame}
%
\begin{frame}
\frametitle{Presentation/lecture: NIME (practices)}
\begin{itemize}
\item Presentation/lecture of a selection of practices in NIME from an HCI perspective + Q\&A.
\end{itemize}
\end{frame}
%
\begin{frame}
\frametitle{Team work: Understanding NIME paper writing}
\begin{itemize}
\item Selection of a paper from the NIME Reader (\url{https://www.springer.com/gp/book/9783319472133}). Discussion about what is ... 
\begin{itemize}
\item the research question (RQ)
\item the approach used to address the RQ / research methods
\item the main findings 
\item the main contribution
\end{itemize}
A high-level summary comparison between the format of NIME and CHI papers is acknowledged.
\end{itemize}
\end{frame}
%
\begin{frame}
\frametitle{Team work: Summaries}
\begin{itemize}
\item The teams summarize to the group their selected paper (10 min per group). 
\begin{itemize}
\item research question (RQ)
\item approach to address the RQ / research methods
\item main findings 
\item main contribution
\end{itemize}
\item Comparison with CHI papers: what are the similarities and differences? 
\end{itemize}
\end{frame}
%
\begin{frame}
\frametitle{Resources}
\begin{itemize}
\item The content of this class can be found on Canvas here:\\
\url{https://uio.instructure.com/courses/11472/pages/human-computer-interaction-1c}
\item The slides of this class can be found on GitHub here: \\
\url{https://github.com/axambo/hci-lecture-slides/tree/master/slides/d3/}
\item Archive of NIME Proceedings:\\
\url{http://www.nime.org/archives/}
\end{itemize}
\end{frame}
%
%\begin{frame}
%  \frametitle{References}
%  \printbibliography
%\end{frame}
%
\end{document}
